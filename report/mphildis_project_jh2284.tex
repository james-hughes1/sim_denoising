\documentclass[12pt]{article}

\title{Data Analysis Project Report}
\author{James Hughes}

\usepackage[nottoc,numbib]{tocbibind}
\usepackage{graphicx}

\begin{document}

\begin{titlepage}
    \begin{center}
        \vspace*{1cm}

        \Huge
        \textbf{Deep Learning for Structured Illumination Microscopy Image Processing}

        \vspace{0.5cm}
        \LARGE

        James Hughes

        Supervised by Dr Edward Ward

        \vspace{2cm}
        \Huge
        \textbf{Project Report}

        \vfill

        MPhil, Data Intensive Science

        \vspace{0.8cm}

        \Large
        Department of Physics \& Department of Chemical Engineering and Biotechnology

        University of Cambridge

        United Kingdom

        28th June 2024

    \end{center}
\end{titlepage}

\pagenumbering{roman}

\newpage
\section*{Acknowledgements}
\addcontentsline{toc}{section}{\protect\numberline{}Acknowledgements}

Firstly I would like to thank my supervisor, Dr Edward Ward, for all of his support over the course of this project.
The project involved a lot of concepts from microscopy and image processing that were very new to me,
but Dr Ward made it clear from early on in the project that this would not be a problem,
and was quick to provide reading materials to help me get to grips with the subject.
Having been trained in Mathematics during my time as an undergraduate,
the chance to go into the laboratory and capture real microscope images that were later used in the work was incredibly exciting.
Dr Ward was eager to provide this opportunity and welcomed me to the Chemical Engineering and Biotechnology (CEB) Department and his research group.

I also had the opportunity to attend some of the Laser Analytics Group (LAG) lab meetings,
where I had the privilege of learning about some of the world-leading research being undertaken by the group.
Later, I shared details about my own project in two presentations to the group.
I would like to thank all of the members of the LAG for welcoming me, listening to my presentations and providing great feedback.
In particular I wish to thank Professor Clemens Kaminski for his helpful suggestions and words of encouragement.

I would also like to thank Jeremy Wilkinson, Esther Gray, and Emilio Luz-Ricca.
I had very insightful conversations about the project with all of them that helped me see the work in a new light.

Lastly, I would like to thank my parents for being a continual source of support and strength throughout my education,
in particular for encouraging me to make the most of every opportunity that comes my way.

\newpage
\begin{abstract}
    \addcontentsline{toc}{section}{\protect\numberline{}Abstract}

\end{abstract}

\newpage
\tableofcontents

\newpage
\pagenumbering{arabic}
\section{Introduction}

Fluorescence microscopy is an essential tool for microbiologists,
enabling them to view complex biological phenomena unfolding at the sub-cellular level.
This type of microscopy is particularly suited to the work of microbiology as it uses dyes to mark specific organic compounds of interest,
which then release photons in response to illumination from a laser at a suitable wavelength.
As a type of optical microscopy, the resolution of these systems is limited by the effects of diffraction.
This limit was quantified by Abbe in 1882 as a minimal resolvable distance

\[\frac{\lambda}{2n\sin\theta}\]

Axial resolution is also an issue.
For comparison, the soma (body) of a human neuron cell is approximately 10-20\textmu m in diameter.
This represents a serious obstacle to researchers attempting to view cell dynamics in greater detail.

Structured Illumination Microscopy (SIM) is a technique that combines a specialised microscope set-up,
alongside computational processing of the acquired images,
in order to surpass the classical Abbe diffraction limit.
The theoretical foundations of the technique were first established in 2008 \cite{originalSIM},
but since then there have been a range of improvements made to the technique.
While SIM does not necessarily provide the greatest improvements in resolution compared to other methods such as confocal,
it has other advantages for researchers interested specifically in capturing imagery of dynamic biological processes over extended periods.
This relates primarily to the issue of phototoxicity effects.
Every time a fluorescence microscopy image of a cell sample is taken, the cell itself is bleached and damaged in the process.
This is particularly troublesome when one wishes to view dynamic processes in live cells,
because the very process of imaging has an effect on the process being captured,
thereby limiting the duration of imagery that can be obtained that is faithful to the true process.
SIM offers a trade-off between resolution improvements and low photo-toxicity effects.

The paper by Li. et al. \cite{keypaper} attempts to augment the SIM image processing pipeline with deep-learning techniques to improve this trade-off.
Their research explores multiple ways in which hardware and computation can be used to improve the resolution of SIM imaging.
This project investigates their `two-step denoising method'.
The core of this technique involves dramatically lowering the illumination dose of the SIM laser,
in order to mitigate phototoxicity effects.
In turn, they train two networks to denoise the acquired and reconstructed images,
to attempt to compensate for the low intensity illumination and reclaim lost image resolution.

The objectives for this project were

reproduce
use good practice to create a robust code base

\section{Methods}

\subsection{SIM Reconstruction process}

Structured Illumination Microscopy stands in contrast to the conventional approach of using a uniform illumination to produce an image.
Instead, SIM microscopes usually employ a spatial light modulator (SLM) to produce a striped illumination pattern,
whose spacing is close to the Abbe diffraction limit of resolution.
When the light illuminates the sample causing it to fluoresce,
the excitation pattern interferes with the high spatial frequencies of the structures in the sample,

\begin{figure}[hbt]
    \includegraphics[scale=0.5]{figures/moire.png}
    \caption{Moir\'{e} Fringes}
    \label{plan2}
\end{figure}

fairSIM and parameters.

\subsection{Data}

In the research Li. et al. acquired pairs of high and low SNR images which were used to train the networks.
In order to avoid to acquire these images more quickly, and to avoid the need for image registration to align the pairs of images acquired,
the increased noise resulting from lowering the illumination dose was simulated in-silico.

2D real data (specifications of the Microscope, cells)

3D vh data
This was generated with the help of another student's code.
The code was modified to include only the most essential parts, and to generate a 3D volume of SIM acquisition stacks (15 each)

\subsection{RCAN}
Use elsewhere

Diagram

\subsection{Pipeline}
Describe building from scratch (pytorch vs tflow)

Diagram

Software

Using CSD3, hardware, parallel -> serial

\section{Results}

\subsection{2D Data}

Parameter estimation

Generalizability

Tables of results, metrics

Images

\subsection{3D Data}

Axial resolution

Tables of results, metrics

Images

\section{Discussion}
Results/conclusions
Further work
What I learned
How I could have improved
- Pt about training second step denoising. Maybe you should have train,test,val,train2,test2,val2.
Otherwise, the step 2 is trained to map denoised images (that step1 has seen and so does better on) to GT,
but then evaluated on how it maps unseen step 1 denoised images to GT. Also the testing set gets seen too much (Mike's
last image analysis lecture about over-exposure to hold out test set)

\bibliographystyle{IEEEtran}
\bibliography{Biblio}

\appendix

\section{Statement on the use of auto-generation tools}

\section {High-Performance Computing Resources}

This work was performed using resources provided by the Cambridge Service for Data Driven Discovery (CSD3) operated by the University of Cambridge Research Computing Service (www.csd3.cam.ac.uk),
provided by Dell EMC and Intel using Tier-2 funding from the Engineering and Physical Sciences Research Council (capital grant EP/T022159/1),
and DiRAC funding from the Science and Technology Facilities Council (www.dirac.ac.uk).

\end{document}
